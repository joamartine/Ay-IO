\section{Diferenciación Horizontal - Hotelling}

Considere una ciudad lineal que va de $0$ a $1$, dos empresas $L$ y $R$ deciden en que parte ubicarse, $\delta _L$ y $\delta_R$ respectivamente. Los potenciales consumidores se distribuyen de forma uniforme y los caracteriza la siguiente función de utilidad,
\begin{equation*}
    U_{ij} = \bar{u} + (y-p_j) - \theta (\delta_j - v_i)^2 \label{función utilidad}
\end{equation*}
Donde $\bar{u}$ es una utilidad 