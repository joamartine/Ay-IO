\section{Bertrand con restricciones de capacidad}

Suponga un mercado con dos firmas que compiten en precios, ofreciendo un producto homogéneo. Considere además que la demanda de mercado corresponde a $Q(p) = 90 -3p$, y las funciones de costo de las firmas son idénticas e iguales a $C(q) = 5q$. Finalmente, es ampliamente conocido que las firmas poseen una capacidad máxima $k$ a producir.

\begin{itemize}
    \item [\textbf{a.-}] ¿Qué sucede si $k\geq 75$? ¿Qué sucede en caso de que $k<75$?
    \begin{solution}
        Dado que en equilibrio tenemos $p = c = 5$, entonces la demanda será 
        \begin{equation*}
            Q(p= 5) = 90-15 - 3\cdot 5 = 75
        \end{equation*}
        Por lo que si $k\geq 75$ cada firma es capaz de bastecer a todo el mercado y se genera competencia a la Bertrand. Si $k<75$, una sola firma no es capaz de bastecer a todo el mercado y habrá espacio para actuar sobre la residual.
    \end{solution}
    \item [\textbf{b.-}] Ahora considere la situación en que $k = 15$. ¿Qué sucede con las firmas en el mercado?
    \begin{solution}
        En este tipo de casos con capacidad simétrica las empresas se adelantarán a lo que hará la otra firma. Desde el punto de vista de la firma $1$ se asume que la firma $2$ vende toda su capacidad ($k =15$) a un precio menor, por supuesto de racionamiento eficiente los más dispuestos a pagar compraran esta oferta. Finalmente tendremos una demanda residual a la que la firma $1$ tendrá que maximizar beneficios. Matemáticamente esto es,  
        \begin{equation*}
            Q^r = 90-k-3p = 75-3p
        \end{equation*}
        Por lo tanto, la maximización de beneficios está dada por
        \begin{align*}
            \max_{p_1} \quad &\Pi_1 = (p-5)(75-3p) \\
            \textbf{CPO:}\quad &\frac{\partial \Pi_1}{\partial p_1} = 75 -6p +15 = 0 \\
            &p = 15
        \end{align*}
        Por lo tanto el $Q^r$ será $75-3\cdot 15 = 30$. \textbf{Sin embargo}, esto no es factible, como dice el enunciado $k=15<30$ a pesar de que la demanda sea $30$ no se puede satisfacer por completo. Dado que no se puede ofrecer $30$, se tendrá que ofrecer lo máximo: $15$, por lo que lo racional sería subir el precio hasta que sólo se vendan $15$ unidades.
        
        Debemos encontrar un precio de equilibrio donde la cantidad demandada del residual sea igual a $15$.
        \begin{align*}
            75-3p= 15 \\
            p = 20
        \end{align*}

        Finalmente podemos decir que una de las empresas ofreció lo máximo que se podía $k = 15$, dada la demanda residual $Q^r = 75-3p$ se ofrece lo máximo que se puede $k =15$ a un precio $20$. 

        Acabamos de determinar la producción y precio de la firma $1$ dado que la firma $2$ vendió toda su capacidad. Por el otro lado \textbf{la firma $2$ va a hacer el mismo razonamiento}, pensando que la firma $1$ venderá toda su capacidad a un precio menor. Por tanto se concluye que \textbf{las dos firmas venden $15$ a un precio $20$. }
    \end{solution}
    \item [\textbf{c.-}] ¿Qué ocurriría en caso de que las empresas pudieran primero elegir sus capacidades y luego competir por precios?
    \begin{solution}
        Si las empresas eligen primeros sus capacidades y luego sus precios, las empresas elegirían capacidad iguales a las cantidad de Cournot y precios iguales al precio de mercado con competencia a la Cournot. Esto implica que: con capacidad (un supuesto muy realista es que las empresas siempre tienen restricciones a la capacidad), los precios están por encima del coste marginal y las empresas ganan beneficios positivos. En realidad, lo que observamos en el mercado es idéntico a lo que observaríamos si compitieran en cantidades: suponer que las empresas elegían cantidades, no era algo tan erróneo como parecía inicialmente. 
        \begin{itemize}
            \item Cuando hay restricciones de capacidad se suaviza la competencia. Los precios de equilibrio no son tan bajos y tenemos que $p> c$ y las empresas tienen beneficios postivios
            \item Las empresas evitan acumular demasiada capacidad para suavizar la competencia en precios, es como un compromiso de que no bajarán mucho los precios.
            \item Ejemplos en dode la elección de capacidad es importante: Hoteles, líneas aéreas
            \item El restultado de juego en 2 etapas coincide con el de Cournot si las capacidades son interpretadas como cantidades.
            \item \textbf{\href{https://www.eco.uc3m.es/~mmachado/Teaching/Industrial2007_2008/3.5.Competenciaenpreciosconrestriccionesdecapacidad_new.pdf}{\underline{Fuente}}} 
            \item \textbf{\href{https://www.eco.uc3m.es/~mmachado/Teaching/Industrial2007_2008/Industrial2006_07/3.4.Competenciaenpreciosconrestriccionesdecapacidad.pdf}{\underline{Más información}}}  
        \end{itemize}
    \end{solution}
\end{itemize}