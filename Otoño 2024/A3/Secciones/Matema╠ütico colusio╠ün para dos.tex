\section{Colusión en Bertrand y Cournot}
Suponga que en China debido al coronavirus solo han quedado dos empresas que comercian animales exóticos para consumo. Ambas empresas tienen los mismos costos marginales, iguales a $c$, y venden un producto homogéneo. La demanda inversa de mercado que enfrentan está dada por $P=A-Q$. Las firmas están estudiando la posibilidad de coludirse en diferentes escenarios, para ello consideremos que las firmas descuentan los beneficios futuros a un factor $\delta$ y ante un desvío aplican la estrategia gatillo. Con esto se le pide que responda lo siguiente:

\begin{enumerate}
  \item Derive la condición que debe cumplir $\delta$ para que la colusión sea sostenible y encuentre el valor del factor $\delta^{C}$ que hace posible la colusión si estas firmas compiten en cantidades, y el factor $\delta^{B}$ que hace posible la colusión si estas firmas compiten en precios. Considere que al coludirse se reparten los beneficios equitativamente.
  \begin{solution}
    Para que una estrategia colusiva sea sostenible, se debe cumplir que los beneficios de la colusión sea mayores o iguales a los de desvío y competencia.
    \begin{align*}
        VP(\text { colusión }) \geq VP(\text { desvío }) \\
        \sum_{i=0}^{\infty} \delta^{i} \pi^{C} \geq \pi^{D}+\sum_{i=1}^{\infty} \delta^{i} \pi^{N}
    \end{align*}
    Aplicamos la suma geométrica para llegar a un delta que cumpla la condición de colusión.
    \begin{align*}
        \frac{\pi^{C}}{1-\delta} &\geq \pi^{D}+\delta \frac{\pi^{N}}{1-\delta} \\
        \pi^{C} &\geq \pi^{D}(1-\delta)+\delta \pi^{N} \\
        \delta\left(\pi^{D}-\pi^{N}\right) &\geq \pi^{D}-\pi^{C} \\
        \delta &\geq \frac{\pi^{D}-\pi^{C}}{\pi^{D}-\pi^{N}}
    \end{align*}
    Solo hace falta calcular los pagos para cada caso y así encontrar el $\delta$ mínimo que cumpla la condición. Los pagos dependerán de la competencia así que vamos caso por caso.

    \rule{\linewidth}{0.4pt} 
    \textbf{Bertrand} 
    
    Los beneficios de coludirse bajo competencia en precios son los beneficios monopólicos repartidos en ambas firmas.
    \begin{align*}
        \pi^{C}= \frac{\pi^{M}}{2}= \frac{(A-c)^{2}}{8}
    \end{align*}
    Los beneficios de desvío son iguales a los beneficios monopólicos. 
    \begin{align*}
        \pi^{D}=\pi^{M}=\frac{(A-c)^{2}}{4}
    \end{align*}
    Y los beneficios de no cooperar son $\pi^{N}=0$. Luego, reemplazando en la condición de colusión.
    \begin{align*}
        \delta^{B}=\frac{\pi^{D}-\pi^{C}}{\pi^{D}-\pi^{N}}=\frac{\frac{(A-c)^{2}}{4}-\frac{(A-c)^{2}}{8}}{\frac{(A-c)^{2}}{4}-0}=\frac{\frac{(A-c)^{2}}{8}}{\frac{(A-c)^{2}}{4}}=\frac{1}{2} \\
        \delta^{B}=\frac{1}{2}
    \end{align*}

    \rule{\linewidth}{0.4pt} 
    \textbf{Cournot}
    
    Los beneficios colusivos son iguales al caso de competencia en precios.
    \begin{align*}
        \pi^{C}=\frac{\pi^{M}}{2}=\frac{(A-c)^{2}}{8}
    \end{align*}
    Luego, los beneficios de competir los obtenemos de la siguiente forma. Primero, de la maximización de la firma $i$, obtenemos la función de reacción de la firma $i$.
    \begin{equation*}
        q_{i}=\frac{A-q_{j}-c}{2}
    \end{equation*} 
    Luego, la cantidad que produce cada firma es $q_{i}=\frac{A-c}{3}$, la cantidad total $Q=\frac{2(A-c)}{3}$ y el precio $P=\frac{A+2 c}{3}$. Entonces los beneficios no colusivos son
    \begin{equation*}
        \pi^{N}=\frac{(A-c)^{2}}{9}
    \end{equation*} Para calcular los beneficios de desvío reemplazamos la cantidad colusiva de la firma $i$ en la función de reacción de la firma $j$.
    \begin{align*}
        q_{j}^{D}=\frac{A-q_{i}^{C}-c}{2}=\frac{A-\frac{A-c}{4}-c}{2}=\frac{3(A-c)}{8}
    \end{align*}
    Los beneficios de desviarse son
    \begin{align*}
        \pi^{D}=\frac{9(A-c)^{2}}{64}
    \end{align*}
    Finalmente reemplazamos en la condición de colusión.
    \begin{align*}
        \delta^{C}=\frac{\pi^{D}-\pi^{C}}{\pi^{C}-\pi^{N}}=\frac{\frac{9(A-c)^{2}}{64}-\frac{(A-c)^{2}}{8}}{\frac{9(A-c)^{2}}{64}-\frac{(A-c)^{2}}{9}} \\
        \delta^{C} \geq \frac{9}{17} \approx 0,53
    \end{align*}
  \end{solution}
    \item ¿Bajo que tipo de competencia es más factible la colusión?

    \begin{solution}
    De la parte anterior se tiene que $\delta^{C}>\delta^{B}$ por lo tanto es más fácil coludirse en Bertrand debido a que el castigo es más severo que en Cournot.
    \end{solution}
\end{enumerate}
