
\section{Propuesto: Colusión con firmas asimétricas}
Un mercado posee una demanda $Q(P)=36-P$. Existen dos empresas que compiten en él mediante precios. La primera tiene costo marginal $c_{1}=0$, mientras que la segunda tiene costo $c_{2}=4$.

\begin{enumerate}
  \item Suponga que las empresas desean coludirse. Cuál será el precio de colusión que escogerían y por qué.
  \item ¿Cuál es el máximo reparto del mercado $S_{2}$ (\%) que podría llevarse la firma 2 para que el acuerdo sea factible, si el factor de descuento intertemporal es $\delta=0,75$?
  \item Obtenga las condiciones para que el acuerdo colusivo sea sostenible si las empresas deciden turnarse la producción. Es decir un periodo solo produce una de ellas y en el siguiente período produce la otra y así sucesivamente.\footnote{\textbf{HINT:} Asuma que en $t=0$, la firma 1 parte produciendo, por lo que se puede desviar al tiro y vender en el próximo período.}
\end{enumerate}