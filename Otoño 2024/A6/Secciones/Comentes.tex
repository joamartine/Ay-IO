\section{Comentes}
\begin{itemize}
    \item[a)] Mantener una colusión estable será más fácil a menor frecuencia de operaciones (frecuencia con que las firmas interactúan y fijan los precios).
    \begin{solution}
        \textbf{Falso.} Mientras menor sea la frecuencia con que las firmas se reúnen y fijan los precios, menor será la estabilidad del acuerdo colusivo.
        
        Si la frecuencia con la que las firmas se reúnen es menor, entonces cualquier desvío del acuerdo tarda más en ser descubierto y, por tanto, los castigos demoran más en ser aplicados, lo que favorece el beneficio de desviarse.
    \end{solution}
    \item[b)] Asuma que las firmas compiten en precios. Se puede afirmar que, a mayor asimetría entre firmas (en términos de costos), menor estabilidad del acuerdo colusivo.
    \begin{solution}
        \textbf{Verdadero.} Si las firmas son muy distintas, es poco probable que logren mantener un acuerdo colusivo.
        
        La razón se debe a que la firma más eficiente tiene beneficios positivos aun cuando no se coluda con sus rivales, por lo que estará dispuesta a ser parte del acuerdo solo si se le concede una participación de mercado relativamente alta (mayor a 0,5 ). El tema es que, si se le concede una porción del mercado demasiado alta a la firma más eficiente, entonces será la firma más ineficiente la que no tenga incentivos a formar parte del acuerdo, ya que tendrá una participación de mercado tan baja que le convendrá más desviarse y asumir el posterior castigo que cooperar.
    \end{solution}
    \begin{itemize}

    \item[c)] Lea el siguiente texto y comente.
    
    ``Garantía de precios bajos: Precios bajos todos los días. ¡Si encuentras un precio más bajo, lo igualamos y te damos un $20\%$ de descuento sobre el precio igualado! ¿precio más bajo en otro lugar?, ¡imposible! *Recuerda: la cotización que entregues para la garantía de precios debe ser de un competidor de la misma localidad".

    Lo anterior, corresponde a una estrategia de precios de una empresa en Chile. Analice si dicha práctica podría afectar la sostenibilidad de un eventual acuerdo colusivo entre dicha firma y sus competidores. Justifique su respuesta.
    \begin{solution}
        Esta práctica, efectivamente afecta la sostenibilidad de un acuerdo colusivo. Por un lado, robustece la sostenibilidad, al hacer partícipes a las y los consumidores del monitoreo de los precios de la firma competidora, facilitando su revisión y con ello, aumentando la transparencia del mercado.

        Por otro lado, robustece la sostenibilidad al fortalecer la estrategia de castigo, haciendo menos beneficiosa la estrategia de desvío, ya que, ante cualquier desvío del precio acordado, inmediatamente la otra firma reaccionaría y aplicaría una disminución del precio en un $20 \%$ por debajo del precio de desvío.
    \end{solution}
    \end{itemize}
\end{itemize}